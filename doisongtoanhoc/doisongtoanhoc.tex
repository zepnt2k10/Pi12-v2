\thispagestyle{doisongtoanhocnone}
\pagestyle{doisongtoanhoc}
\everymath{\color{doisongtoanhoc}}
\graphicspath{{../doisongtoanhoc/pic/}}
%\blfootnote{$^1$\color{doisongtoanhoc}Trung tâm Thông tin -- Tư liệu, Viện Hàn lâm Khoa học và Công nghệ Việt Nam.}
\begingroup
\AddToShipoutPicture*{\put(0,616){\includegraphics[width=19.3cm]{../bannerdoisong}}}
\AddToShipoutPicture*{\put(43,527){\includegraphics[scale=0.95]{../tieude.pdf}}}\centering
\endgroup

\vspace*{185pt}


\begin{multicols}{2}	
	Đại hội Toán học Quốc tế (International Congress of Mathematicians, sau đây sẽ viết tắt là ICM) là sự kiện khoa học quan trọng hàng đầu của các nhà toán học trên thế giới. Nhiều người trong chúng ta đã quen thuộc với Kỳ thi Olympic Toán quốc tế (IMO). Vậy ICM có gì giống và khác với IMO? Mục đích của ICM là gì? Những hoạt động chính tại một kỳ ICM là gì? ICM $2022$ có những dấu ấn gì đặc biệt? Chúng tôi sẽ thử giải đáp các câu hỏi này.
	\vskip 0.1cm
	$\pmb{1.}$ \textbf{\color{doisongtoanhoc}ICM có gì giống và khác với IMO?}
	\vskip 0.1cm
	Cũng giống như IMO, ICM là một hoạt động cộng đồng hướng đến mục tiêu phát triển sự quan tâm đến toán học. Có lịch sử lâu đời hơn IMO một chút, ICM đầu tiên được tổ chức từ năm $1897$ tại Z\"urich, Thụy Sĩ, nhưng trong khi IMO được tổ chức hầu như hằng năm, thì ICM được tổ chức bốn năm một lần. Nếu IMO tập trung vào việc giải các bài toán, thì ICM hướng đến trình bày những thành tựu nghiên cứu toán học đáng kể nhất gần đây. Không có sự khác biệt đáng kể giữa nghiên cứu toán học và giải các bài toán Olympic, vì cả hai công việc đều đòi hỏi kỹ năng giải quyết vấn đề. Chúng ta biết rằng có những bài toán mở nổi tiếng trong toán học, như  định lý lớn Fermat (là bài toán mở đến trước $1994$), giả thuyết về số nguyên tố sinh đôi, hay giả thuyết Riemann (cả hai hiện vẫn chưa được giải quyết). Những tiến bộ về các bài toán mở nổi tiếng, nếu có, cũng là một điểm nhấn quan trọng của những kỳ ICM. Nhưng so với việc thi olympic, có thể nói các nhà toán học có nhiều tự do hơn trong việc làm nghiên cứu của mình, họ không nhất thiết phải làm việc với một vấn đề có sẵn. Có những nhà toán học lớn theo đuổi một vấn đề  hàng năm, thậm chí hàng chục năm trời. Việc một nhà số học ngồi nghe một bài giảng hình học đại số, hay một nhà đại số dự một bài giảng vật lý toán, để mở mang kiến thức, cũng là một việc thường xảy ra và được ICM khuyến khích.
	\vskip 0.1cm
	$\pmb{2.}$ \textbf{\color{doisongtoanhoc}Vì sao cần tổ chức ICM?}
	\vskip 0.1cm
	Mục đích chính của ICM là để tạo điều kiện cho các nhà toán học từ khắp nơi trên thế giới gặp gỡ những chuyên gia hàng đầu, và để tôn vinh những thành tựu toán học nổi bật nhất gần đây.
	\vskip 0.1cm
	Các nhà toán học gặp gỡ nhau? Chẳng phải các nhà toán học chỉ cần có giấy bút (và máy tính) để làm việc đó sao? Đúng là phần lớn các nhà toán học  có thiên hướng lý thuyết,  không cần nhiều trang thiết bị để làm việc. Nhưng ngoài giấy bút và máy tính, họ cũng thường cần một người đồng nghiệp ăn ý để thử nghiệm những ý tưởng chợt đến, tranh cãi về một chứng minh trong một bài báo, hay đơn giản tán gẫu về trận bóng tối qua. 
	\vskip 0.1cm
	Như bất cứ ngành khoa học có truyền thống nào, toán học ngày càng đa dạng hóa và chuyên môn hóa cao độ, với rất nhiều phân ngành khác nhau. ICM đầu tiên năm $1897$ chỉ có năm tiểu ban, mỗi tiểu ban phụ trách một chuyên môn, gồm có số học và đại số, giải tích và lý thuyết hàm, hình học, cơ học và vật lý toán, lịch sử và thư mục toán học. Nửa thế kỷ sau, ICM $1958$ mới có tám tiểu ban\footnote{\color{doisongtoanhoc}Xem Guillermo P. Curbera, \emph{Mathematicians of the World, Unite!}, Wellesley, Massachusetts: A.K. Peters, Ltd ($2009$), trang $14$ và $141$.}. Đến ICM $2022$, ta có đến hai mươi tiểu ban: logic, đại số, hình học đại số và hình học phức, tôpô, lý thuyết Lie và các mở rộng, giải tích, động lực học, phương trình vi phân, vật lý toán... Mỗi tiểu ban ngày nay lại có nhiều tiểu mục nhỏ hơn, ví dụ tiểu ban đại số có bốn tiểu mục.  Ngay từ ICM năm $1908$ ở Rome, Poincar\'e đã nhận ra: ``Khi một khoa học càng phát triển, ta càng khó nắm bắt được toàn bộ khoa học đó. Từ đó người ta phải chia nhỏ khoa học ấy ra nhiều phần, và bằng lòng với việc chỉ quan tâm tới đúng một trong các phần đó, nói cách khác là phải chuyên môn hóa. Chuyên môn hóa quá sâu sẽ làm cản trở nghiêm trọng đến tiến bộ chung của khoa học (...) Chính nhờ những tương tác bất ngờ giữa các hướng nghiên cứu khác nhau mà khoa học mới có thể phát triển."\footnote{\color{doisongtoanhoc}``In proportion as the science develops, it becomes more difficult to take it in its entirety. Then an attempt is made to
		cut it in pieces and to be satisfied with one of these pieces -- in
		a word, to specialize. Too great a movement in this direction
		would constitute a serious obstacle to the progress of science. As I have said, it is by unexpected concurrences
		between its different parts that it can make progress." Xem The Mathematical Intelligencer, Tập $34$, số $2$ ($2012$), trang $15-29$.} Nhà toán học nổi tiếng Hoàng Tụy ($1927-2019$) sinh thời thường cảnh báo về nguy cơ của chủ nghĩa tỉnh lẻ, khi một nhà khoa học làm việc trong tinh thần bế quan tỏa cảng và khiến bản thân thui chột. Gặp gỡ và tương tác với những chuyên gia hàng đầu tại những sự kiện lớn như ICM là cách giúp một nhà khoa học ``mở con mắt hướng ra những bờ cõi khoa học nơi những người khác đang chiếm cứ, và buộc ta phải so sánh thành tựu của mình với họ, qua đó nhận ra ngôi làng mình đang sống nhỏ bé chừng nào", theo lời của Poincar\'e\footnote{\color{doisongtoanhoc}Tài liệu đã dẫn, trang $20$.}.
	\vskip 0.1cm
	$\pmb{3.}$ \textbf{\color{doisongtoanhoc}Hoạt động chính ở các ICM là gì?}
	\vskip 0.1cm
	Hoạt động chính của ICM là các bài giảng của các chuyên gia, và phần trao giải thưởng ghi nhận những thành tựu toán học cao nhất đã đạt được giữa hai kỳ đại hội. Các bài giảng của chuyên gia gồm hai loại là các bài giảng toàn thể (khoảng $20$ bài) và các bài giảng tiểu ban (khoảng 180 bài). Được đọc bài giảng tại một ICM là một vinh dự lớn, nếu ta biết rằng mỗi năm có khoảng gần một trăm nghìn công trình toán học được xuất bản trên các tạp chí chuyên ngành. Các giải thưởng được trao ở các kỳ ICM gần đây là huy chương Fields (cho nhà toán học trẻ xuất sắc), giải thưởng Nevanlinna (khía cạnh toán học trong tin học), giải Gauss (toán ứng dụng), huy chương Chern (những người có thành tựu toán học xuất chúng, không hạn chế tuổi tác), và giải Leelavati (phổ biến kiến thức và quảng bá toán học). Trong số này, huy chương Fields nói chung được coi là giải thưởng danh giá nhất.
	\vskip 0.1cm
	$\pmb{4.}$ \textbf{\color{doisongtoanhoc}Ai có thể giành được huy chương Fields?}
	\vskip 0.1cm
	Huy chương Fields được trao cho những nhà toán học dưới $40$ tuổi với thành tựụ xuất sắc và tiềm năng phát triển lớn. Nhìn vào những người đoạt huy chương Fields gần đây, ta thấy họ nói chung là những người tài năng, được đào tạo bài bản (có bằng tiến sĩ), và theo đuổi những lĩnh vực quan trọng hoặc những bài toán quan trọng trong một thời gian dài. Có lẽ cách chắc chắn nhất để \emph{không} giành được giải Fields là lao vào những bài toán nổi tiếng, nhiều người biết, như giả thuyết Collatz, hay giả thuyết Riemann, bằng tay không, không quan tâm đến việc tích lũy kiến thức và các kỹ thuật cơ bản dần dần. Những huy chương Fields, ngoài sự dũng cảm tấn công những vấn đề khó khăn, còn ghi dấu ấn cụ thể, thuyết phục với những bài báo với nhiều người đọc, được bình duyệt chặt chẽ, trên những tạp chí toán học uy tín. Thông thường họ đã có một công chúng rộng lớn, những ý tưởng của họ đã có ích lợi đáng kể cho công việc của nhiều người khác, \emph{trước khi} được vinh danh với huy chương Fields, chứ không phải ngược lại. Tất nhiên không có quy tắc chung cho những tài năng đặc biệt, họ thường  phá vỡ những quy tắc mà chúng ta coi như hiển nhiên.
	\vskip 0.1cm
	$\pmb{5.}$ \textbf{\color{doisongtoanhoc}Đại hội Toán học Quốc tế $2022$ có gì đặc biệt?}
	\vskip 0.1cm
	ICM $2022$ được tổ chức từ ngày mùng $6$ đến $14/7/2022$. Lễ trao các giải thưởng được tổ chức trực tiếp tại Helsinki, Phần Lan vào ngày $5/7$. Về hình thức, sau $125$ năm lịch sử, đây là lần đầu tiên chương trình khoa học của ICM được tổ chức theo hình thức trực tuyến. Tại ICM năm nay, huy chương Fields đã được trao cho Hugo Dominil--Copin (Pháp), June Huh (Hàn Quốc), James Maynard (Vương quốc Anh), và Maryna Viazovska (Ukraina). Đây đều là những tên tuổi hàng đầu của toán học đương đại, những người sẽ còn tiếp tục ảnh hưởng lâu dài đến toán học. Nếu bạn nghĩ một bài giảng toán học là đối cực của một tiểu thuyết hấp dẫn/một bộ phim hay? Mời bạn xem bài giảng hết sức sáng sủa với đoạn cao trào tuyệt vời của June Huh, một người làm toán (những bài toán không hề đơn giản!) với hồn thơ đặc biệt. Nếu bạn, sau khi sở hữu một cỗ máy tiêu diệt hàng loạt các loại bài toán khó muôn hình vạn trạng, nghĩ quảng bá toán học là một việc nhàm chán và không cần công phu gì? Mời bạn xem bài giảng của Nikolai Andreev (giải Leelavati $2022$) và trang mạng độc đáo của ông\footnote{\color{doisongtoanhoc}https://etudes.ru/}.
	\vskip 0.1cm
	Trong một thế giới còn rất nhiều xung đột và đối kháng, các kỳ ICM và Liên đoàn Toán học Quốc tế (IMU) hướng đến gắn kết con người bằng toán học và tinh thần quốc tế, vượt qua những giới hạn về tư tưởng, ngôn ngữ, văn hóa, mặc cảm thượng đẳng/mặc cảm thấp kém thường ám ảnh con người. Từ khi kỳ ICM đầu tiên được tổ chức đến nay, toán học đã gánh chung tác động của hai thế chiến, chiến tranh lạnh, kỳ thị và phân biệt đối xử, với những hoạt động khác của con người. Vượt qua những khó khăn đó, cộng đồng toán học đã tiếp tục đi tới với lý tưởng bình đẳng, nhân văn, và chủ nghĩa quốc tế, di sản to lớn của những nhà toán học hàng đầu trong quá khứ. Chúng ta hy vọng vào thành công trong tương lai của toán học và của những lý tưởng này.
\end{multicols}
